\documentclass[letterpaper,twocolumn,10pt]{article}
\usepackage{usenix}


\usepackage{amsmath}



%-------------------------------------------------------------------------------
\begin{document}
%-------------------------------------------------------------------------------

%don't want date printed
\date{}


\title{\Large \bf My Draft Proposal}

%for single author (just remove % characters)
\author{
{\rm Your N.\ Here}\\
KAUST
\and
{\rm Second Author}\\
KAUST
% copy the following lines to add more authors
% \and
% {\rm Name}\\
%Name Institution
} % end author

\maketitle

%-------------------------------------------------------------------------------
\begin{abstract}
%-------------------------------------------------------------------------------
Your abstract text goes here. The abstract is optional for the 2-page draft proposal.
An effective abstract follows some basic rules, which you can read here \cite{john-wilkes-keynote}.
\end{abstract}

%-------------------------------------------------------------------------------
\section{Project type}
%-------------------------------------------------------------------------------

Declare the project type:\\
This project is original research.\\
OR\\
This project is a reproduction study.

Declare if the project explores/doesn't explore the use of GenAI tools to aid the systems research process.


%-------------------------------------------------------------------------------
\section{Introduction}
%-------------------------------------------------------------------------------

The initial project proposal draft is 2 pages including references that ideally includes:

The particular results you would like to replicate or the overall goal of your original project.

Ensure that the proposal answers these questions:
\begin{itemize}
\item What is the problem?
\item Why is it important to solve?
\item What you will do in some detail?
\item How would you evaluate your solution?
\end{itemize}

\section{Tasks and timeline}

Include a brief outline of incremental steps to do to finish the project as well as a timeline.
\subsection{Task 1}
We plan to do X.
This will take 2 weeks.

\subsection{Task 2}
We plan to do Y.
This will take 3 weeks.

%-------------------------------------------------------------------------------
\section*{Group members}
%-------------------------------------------------------------------------------

This proposal is by:
\begin{itemize}
  \item Teammate 1 - KAUST ID
  \item Teammate 2 - KAUST ID
\end{itemize}

%-------------------------------------------------------------------------------
\section*{Checklist}
%-------------------------------------------------------------------------------

\begin{itemize}
  \item Draft the 2-page proposal (remove the Checklist in the final version of this document)
  \item Ensure the group members are listed, the project type is declared, and the GenAI-aided systems research is declared.
  \item Submit to the instructor; ensure the email subject includes the tag ``[CS345]''
  \item Schedule a 15-minute meeting to discuss
\end{itemize}

%-------------------------------------------------------------------------------
\bibliographystyle{plain}
\bibliography{main.bib}

%%%%%%%%%%%%%%%%%%%%%%%%%%%%%%%%%%%%%%%%%%%%%%%%%%%%%%%%%%%%%%%%%%%%%%%%%%%%%%%%
\end{document}
%%%%%%%%%%%%%%%%%%%%%%%%%%%%%%%%%%%%%%%%%%%%%%%%%%%%%%%%%%%%%%%%%%%%%%%%%%%%%%%%